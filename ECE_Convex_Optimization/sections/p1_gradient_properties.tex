%------------------------------------------------------%
%------------------------------------------------------%
\section{Direction of Maximum Increase}
%------------------------------------------------------%
%------------------------------------------------------%

\subsection{Concept}
\begin{itemize}
	\item A vector \(\boldsymbol{d} \in \mathbb{R}^{n}, \boldsymbol{d} \neq \boldsymbol{0}\), is a feasible direction at \( \boldsymbol{x} \in \Omega\) if there exists \(\alpha_{0}>0\) such that \(\boldsymbol{x}+\alpha \boldsymbol{d} \in \Omega\) for all \( \alpha \in\left[0, \alpha_{0}\right] \).

	\item Let \(f: \mathbb{R}^{n} \rightarrow \mathbb{R}\) be a real-valued function and let \(\boldsymbol{d}\) be a feasible direction at \(\boldsymbol{x} \in \Omega \). The directional derivative of \(f\) in the direction \(\boldsymbol{d}\), denoted \(\partial f / \partial \boldsymbol{d}\) is the real-valued function defined by
	
	\[
	\frac{\partial f}{\partial \boldsymbol{d}}(\boldsymbol{x})=\lim _{\alpha \rightarrow 0} \frac{f(\boldsymbol{x}+\alpha \boldsymbol{d})-f(x)}{\alpha} = \boldsymbol{d}^{\top} \nabla f(\boldsymbol{x})
	= \sum_{i=1}^{n} \frac{\partial f(\boldsymbol{x})}{x_i} d_i
	\]

	\item If Euclidean norm \(\|\boldsymbol{d}\|_2=1\), then \(\partial f / \partial \boldsymbol{d}\) is the \underline{rate of increase} of \(f\) at \(\boldsymbol{x}\) in the direction \(\boldsymbol{d}\).
	
	\[\frac{\partial f}{\partial \boldsymbol{d}}(\boldsymbol{x})=\left.\frac{d}{d \alpha} f(\boldsymbol{x}+\alpha \boldsymbol{d})\right|_{\alpha=0}=\nabla f(\boldsymbol{x})^{\top} \boldsymbol{d}=\langle\nabla f(\boldsymbol{x}), \boldsymbol{d}\rangle=\boldsymbol{d}^{\top} \nabla f(\boldsymbol{x}) \]

	\item If Euclidean norm \(\|\boldsymbol{d}\|_2 \neq 1\), the rate of increase of \(f\) at \(\boldsymbol{x}\) in the direction \(\boldsymbol{d}\), normalize \(\boldsymbol{d}\), replace \(\boldsymbol{d}\) with \(\frac{\boldsymbol{d}}{\|\boldsymbol{d}\|_2}\), that is, \(\frac{\boldsymbol{d}^{\top}}{\|\boldsymbol{d}\|_2} \nabla f(\boldsymbol{x}) \).
	
	\[ \left\| \frac{\boldsymbol{d}}{\|\boldsymbol{d}\|} \right\| 
	= \left\| \frac{1}{\|\boldsymbol{d}\|}  \left[ \begin{array}{c} \\ \boldsymbol{d} \\ \\ \end{array} \right]\right \|
	= \frac{1}{\|\boldsymbol{d}\|}  \|\boldsymbol{d}\|
	= 1 \]
\end{itemize}

%------------------------------------------------------%
\subsection{Practice problems}
\begin{enumerate}
	\item Given the following function,
\end{enumerate}

\begin{equation*}
	f\left(x_{1}, x_{2}\right)=x_{1}^{2} x_{2}+x_{2}^{3} x_{1} .
\end{equation*}

(1) In what direction does the function \(f\) increase most rapidly at the point \(\boldsymbol{x}^{(0)}=\left[\begin{array}{ll}2 & 1\end{array}\right]^{\top}\) ?

(2) What is the rate of increase of \(f\) at the point \(\boldsymbol{x}^{(0)}\) in the direction of maximum increase of \(f\) ?

(3) Find the rate of increase of \(f\) at the point \(\boldsymbol{x}^{(0)}\) in the direction \(\boldsymbol{d}=\left[\begin{array}{ll}4 & 3\end{array}\right]^{\top}\).

\textbf{Short answer:}
(1) A differentiable function \(f\) increases most rapidly in the direction of the gradient. In this problem, we have

\begin{equation*}
	\nabla f(\boldsymbol{x})=\left[\begin{array}{l}
		2 x_{1} x_{2}+x_{2}^{3} \\
		x_{1}^{2}+3 x_{1} x_{2}^{2}
	\end{array}\right] .
\end{equation*}

Hence,

\begin{equation*}
	\nabla f\left(\boldsymbol{x}^{(0)}\right)=\left[\begin{array}{c}
		5 \\
		10
	\end{array}\right]
\end{equation*}

(2) The rate of increase of \(f\) at \(\boldsymbol{x}^{(0)}\) in the direction \(\nabla f\left(\boldsymbol{x}^{(0)}\right)\) is

\begin{equation*}
	\nabla f\left(\boldsymbol{x}^{(0)}\right)^{\top} \frac{\nabla f\left(\boldsymbol{x}^{(0)}\right)}{\left\|\nabla f\left(\boldsymbol{x}^{(0)}\right)\right\|}=\left\|\nabla f\left(\boldsymbol{x}^{(0)}\right)\right\|=5 \sqrt{5} .
\end{equation*}

(3) The rate of increase of \(f\) at \(\boldsymbol{x}^{(0)}\) in the direction \(\boldsymbol{d}\) is

\begin{equation*}
	\nabla f\left(\boldsymbol{x}^{(0)}\right)^{\top} \frac{\boldsymbol{d}}{\|\boldsymbol{d}\|}=\left[\begin{array}{ll}
		5 & 10
	\end{array}\right]\left[\begin{array}{l}
		4 \\
		3
	\end{array}\right] \frac{1}{5}=10 .
\end{equation*}

%------------------------------------------------------%
\bigskip
\noindent
\begin{enumerate}
	\setcounter{enumi}{1}
	\item Find the range of values of the parameter \(a\) for which \(\boldsymbol{d}=\left[\begin{array}{ll}a & 1\end{array}\right]^{\top}\) is a direction of ascent of
\end{enumerate}

\begin{equation*}
	f=f\left(x_{1}, x_{2}\right)=x_{1}^{3}+x_{1} x_{2}-x_{1}^{2} x_{2}^{2},
\end{equation*}

at the point \(\boldsymbol{x}^{(0)}=\left[\begin{array}{ll}1 & 1\end{array}\right]^{\top}\).


\textbf{Short answer:}
For the direction of ascent, we have \(\boldsymbol{d}^{\top} \nabla f>0\). We first calculate the gradient

\begin{equation*}
	\nabla f(\boldsymbol{x})=\left[\begin{array}{c}
		3 x_{1}^{2}+x_{2}-2 x_{1} x_{2}^{2} \\
		x_{1}-2 x_{1}^{2} x_{2}
	\end{array}\right] .
\end{equation*}

Hence,

\begin{equation*}
	\nabla f\left(\boldsymbol{x}^{(0)}\right)=\left[\begin{array}{c}
		2 \\
		-1
	\end{array}\right]
\end{equation*}

We then need

\begin{equation*}
	\boldsymbol{d}^{\top} \nabla f\left(\boldsymbol{x}^{(0)}\right)=\left[\begin{array}{ll}
		a & 1
	\end{array}\right]\left[\begin{array}{c}
		2 \\
		-1
	\end{array}\right]=2 a-1>0 .
\end{equation*}

We have \(a>\frac{1}{2}\).
