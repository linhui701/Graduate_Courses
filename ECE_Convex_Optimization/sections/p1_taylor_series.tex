%------------------------------------------------------%
%------------------------------------------------------%
\section{Taylor Series Expansion}
%------------------------------------------------------%
%------------------------------------------------------%

%------------------------------------------------------%
\subsection{Concept}
\begin{itemize}
	\item Suppose \(f \in \mathcal{C}^{2}\). The Taylor series expansion of a real-valued function \(f\) : \(\mathbb{R}^{n} \rightarrow \mathbb{R}\) about the point \(\boldsymbol{x}_{0} \in \mathbb{R}^{n}\) is
\end{itemize}

\begin{equation*}
	\begin{aligned}
		f(\boldsymbol{x})=f\left(\boldsymbol{x}_{0}\right) 
		& + \left[ \begin{array}{ccc}
			& D f\left(\boldsymbol{x}_{0}\right) & \\
			\end{array}
			\right]
			\left[ \begin{array}{c} 
				 \\ \left(\boldsymbol{x}-\boldsymbol{x}_{0}\right) \\ \\
			\end{array}
			\right] \\
		& +\frac{1}{2}
			\left[ \begin{array}{ccc} 
				& \left(\boldsymbol{x}-\boldsymbol{x}_{0}\right)^{\top}  & \\
			\end{array}
			\right]
			\left[ \begin{array}{ccc}
				& & \\
				& D^2 f\left(\boldsymbol{x}_{0}\right) & \\
				& & \\
			\end{array}
			\right]
			\left[ \begin{array}{c} 
				\\ \left(\boldsymbol{x}-\boldsymbol{x}_{0}\right) \\ \\
			\end{array}
			\right]
			+o\left(\left\|\boldsymbol{x}-\boldsymbol{x}_{0}\right\|^{2}\right) .
	\end{aligned}
\end{equation*}

\begin{itemize}
	\item The linear approximation of \(f\) about the point \(\boldsymbol{x}_{0}\) is
\end{itemize}

\begin{equation*}
	l(\boldsymbol{x})=f\left(\boldsymbol{x}_{0}\right)+D f\left(\boldsymbol{x}_{0}\right)\left(\boldsymbol{x}-\boldsymbol{x}_{0}\right) .
\end{equation*}

\begin{itemize}
	\item The quadratic approximation of \(f\) about the point \(\boldsymbol{x}_{0}\) is
\end{itemize}

\begin{equation*}
	q(\boldsymbol{x})=f\left(\boldsymbol{x}_{0}\right)+D f\left(\boldsymbol{x}_{0}\right)\left(\boldsymbol{x}-\boldsymbol{x}_{0}\right)+\frac{1}{2}\left(\boldsymbol{x}-\boldsymbol{x}_{0}\right)^{\top} D^{2} f\left(\boldsymbol{x}_{0}\right)\left(\boldsymbol{x}-\boldsymbol{x}_{0}\right) .
\end{equation*}

%------------------------------------------------------%
\subsection{Examples}

\begin{enumerate}
	\item Compute the linear \(l\left(x_{1}, x_{2}\right)\) and quadratic \(q\left(x_{1}, x_{2}\right)\) approximations of the function \(f\left(x_{1}, x_{2}\right)=x_{1}+\frac{3 x_{2}}{x_{1}}\), at the point \(\boldsymbol{x}^{(0)}=\left[\begin{array}{ll}1 & 2\end{array}\right]^{\top}\).
\end{enumerate}

\textbf{Short answer:}

(1) \[l(\boldsymbol{x})=f\left(\boldsymbol{x}^{(0)}\right)+D f\left(\boldsymbol{x}^{(0)}\right)\left(\boldsymbol{x}-\boldsymbol{x}^{(0)}\right),\]

\begin{equation*}
	D f(\boldsymbol{x}^{(0)})
	=\left[\begin{array}{ll}
		\frac{\partial f}{\partial x_{1}} & \frac{\partial f}{\partial x_{2}}
		\end{array}
		\right]
	=\left[\begin{array}{ll}
		1-\frac{3 x_{2}}{x_{1}^{2}} & \frac{3}{x_{1}}
		\end{array}
		\right]
	=\left[\begin{array}{ll}
		-5 & 3
	\end{array}\right].
\end{equation*}

Therefore,

\begin{equation*}
	l(\boldsymbol{x})=7+\left[\begin{array}{ll}
		-5 & 3
	\end{array}\right]\left[\begin{array}{l}
		x_{1}-1 \\
		x_{2}-2
	\end{array}\right]=-5 x_{1}+3 x_{2}+6 .
\end{equation*}

(2)
\[q(\boldsymbol{x})=l(\boldsymbol{x})+\frac{1}{2}\left(\boldsymbol{x}-\boldsymbol{x}_{0}\right)^{\top} D^{2} f\left(\boldsymbol{x}_{0}\right)\left(\boldsymbol{x}-\boldsymbol{x}_{0}\right). \]

\begin{equation*}
	F(\boldsymbol{x}^{(0)})
	=D^{2} f(\boldsymbol{x})
	=\left[\begin{array}{cc}
		\frac{6 x_{2}}{x_{1}^{3}} & -\frac{3}{x_{1}^{2}} \\
		-\frac{3}{x_{1}^{2}} & 0
	\end{array} 
	\right]
	= \left[\begin{array}{cc}12 & -3 \\ -3 & 0\end{array}\right].
\end{equation*}

Therefore,

\begin{equation*}
	\begin{aligned}
		q(\boldsymbol{x}) & =l(\boldsymbol{x})+\frac{1}{2}\left[\begin{array}{ll}
			x_{1}-1 & x_{2}-2
		\end{array}\right]\left[\begin{array}{cc}
			12 & -3 \\
			-3 & 0
		\end{array}\right]\left[\begin{array}{l}
			x_{1}-1 \\
			x_{2}-2
		\end{array}\right] \\
		& =6 x_{1}^{2}-3 x_{1} x_{2}-11 x_{1}+6 x_{2}+6 .
	\end{aligned}
\end{equation*}


%------------------------------------------------------%
\bigskip
\noindent
\begin{enumerate}
	\setcounter{enumi}{1}
	\item Perform a second-order Taylor series expansion of the function
	
	\begin{equation*}
		f=f\left(x_{1}, x_{2}\right)=3 x_{1}^{2}x_{2}+x_{1} x_{2}^{4}-5 x_{1}+7,
	\end{equation*}
	
	at the point \(\boldsymbol{x}^{(0)}=\left[\begin{array}{ll}0 & 1\end{array}\right]^{\top}\).
\end{enumerate}
	
\textbf{Short answer:}

\begin{equation*}
	\begin{aligned}
		f(\boldsymbol{x}) & =f\left(\boldsymbol{x}_{0}\right)+D f\left(\boldsymbol{x}_{0}\right)\left(\boldsymbol{x}-\boldsymbol{x}_{0}\right) \\
		 	  & \hspace{1.6cm} +\frac{1}{2}\left(\boldsymbol{x}-\boldsymbol{x}_{0}\right)^{\top} 
		 	  	D^{2} f\left(\boldsymbol{x}_{0}\right)\left(\boldsymbol{x}-\boldsymbol{x}_{0}\right)+o\left(\left\|\boldsymbol{x}-\boldsymbol{x}_{0}\right\|^{2}\right) . \\
		Df(\boldsymbol{x}) & =\nabla f(\boldsymbol{x})^{\top}
			= \left[\begin{array}{c}
			6 x_{1} x_{2}+x_{2}^{4}-5 \\
			3 x_{1}^{2}+4 x_{1} x_{2}^{3}
			\end{array}\right]^{\top} . \\
		F(\boldsymbol{x}) & =D^{2} f(\boldsymbol{x})=
			\left[\begin{array}{cc}
			6 x_{2} & 6 x_{1}+4 x_{2}^{3} \\
			6 x_{1}+4 x_{2}^{3} & 12 x_{1} x_{2}^{2}
			\end{array}\right] .
	\end{aligned}
\end{equation*}

We have \(D f\left(\boldsymbol{x}^{(0)}\right)=\left[\begin{array}{ll}-4 & 0\end{array}\right]\), and \(F\left(\boldsymbol{x}^{(0)}\right)=\left[\begin{array}{ll}6 & 4 \\ 4 & 0\end{array}\right]\). Therefore,

\begin{equation*}
	\begin{aligned}
	q(\boldsymbol{x})
		  & = 3(0)^{2}(1)+(0)(1)^{4}-5(0)+7 
		  		+ \left[\begin{array}{cc} -4 & 0 \\ \end{array}\right] 
		  		\left[\begin{array}{c} x_{1}-0 \\ x_{2}-1 \end{array}\right]\\
		  &	\qquad +  \frac{1}{2} \left[\begin{array}{c} x_{1}-0 \\ x_{2}-1 \end{array}\right]^{\top}
		  		\left[\begin{array}{ll}6 & 4 \\ 4 & 0\end{array}\right]
		  		\left[\begin{array}{c} x_{1}-0 \\ x_{2}-1 \end{array}\right] \\
		  & = 7 - 4x_{1} + \left[\begin{array}{cccc} 3x_{1}+2x_{2} -2 & & 2x_{1} \end{array}\right]
		  		\left[\begin{array}{c} x_{1}-0 \\ x_{2}-1 \end{array}\right] \\
		  & =3 x_{1}^{2}+4 x_{1} x_{2}-8 x_{1}+7. \\
		 f(\boldsymbol{x})
		 & =q(\boldsymbol{x})+o\left(\left\|\boldsymbol{x}-\boldsymbol{x}_{0}\right\|^{2}\right) \\
		 & =3 x_{1}^{2}+4 x_{1} x_{2}-8 x_{1}+7 +o\left(\left\|\boldsymbol{x}-\boldsymbol{x}_{0}\right\|^{2}\right).
	\end{aligned}
\end{equation*}

\bigskip

\noindent
[Ref]: Edwin K.P. Chong, Stanislaw H. Żak, ``PART I MATHEMATICAL REVIEW" in ``An introduction to optimization", 4th Edition, John Wiley and Sons, Inc. 2013.
