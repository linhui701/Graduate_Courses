%------------------------------------------------------%
%------------------------------------------------------%
\section{Linear Programming and its Duality}
%------------------------------------------------------%
%------------------------------------------------------%

%------------------------------------------------------%
\subsection{Practice problems}

Example 1
\[
	\begin{array}{rl}
		\operatorname{maximize} & 5 x_{1}+2 x_{2} \vspace{2mm}\\
		\text { subject to} & x_{1}+x_{2}+x_{3}=5 \vspace{2mm}\\
		& -3 x_{1}+2 x_{2}-x_{4}=-5 \vspace{2mm}\\
		& x_{i} \geq 0, \quad i=1,2,3,4 .
	\end{array}
\]

\textbf{Short answer}:
We transform the problem into the following equivalent problem,

\[
	\begin{array}{rl}
		\operatorname{minimize} & -5 x_{1}-2 x_{2} \vspace{2mm}\\
		\text { subject to} & x_{1}+x_{2}+x_{3}=5 \vspace{2mm}\\
		& 3 x_{1}-2 x_{2}+x_{4}=5 \vspace{2mm}\\
		& x_{i} \geq 0, i=1,2,3,4 .
	\end{array}
\]

We form the tableau for the problem and process with simplex procedure.

\begin{equation*}
	\begin{array}{cccc|c}
		1 & 1 & 1 & 0 & 5 \\
		3 & -2 & 0 & 1 & 5 \\
		\hline-5 & -2 & 0 & 0 & 0
	\end{array}
\end{equation*}

We first pivot about \((2,1)\) th element, to get

\begin{equation*}
	\begin{array}{cccc|c}
		0 & 5 / 3 & 1 & -1 / 3 & 10 / 3 \\
		1 & -2 / 3 & 0 & 1 / 3 & 5 / 3 \\
		\hline 0 & -16 / 3 & 0 & 5 / 3 & 25 / 3
	\end{array}
\end{equation*}

\begin{center}
	\begin{tabular}{cccc|c}
		0 & 1 & \(3 / 5\) & \(-1 / 5\) & 2 \\
		1 & 0 & \(2 / 5\) & \(1 / 5\) & 3 \\
		\hline
		0 & 0 & \(16 / 5\) & \(3 / 5\) & 19 \\
		\hline
	\end{tabular}
\end{center}

We then pivot about \((1,2)\) th element, to get the tableau above.

We have the optimized basic feasible solution as \(\boldsymbol{x}=\left[\begin{array}{llll}3 & 2 & 0 & 0\end{array}\right]^{\top}\). The optimal value of \(5 x_{1}+2 x_{2}\) is 19 .


\medskip
\noindent
Example 2. Find the dual problem corresponding to the following primal problem:

\[
	\begin{array}{rl}
		\operatorname{minimize} & c^{\top} x \vspace{2mm}\\
		\text { subject to} & A_{1} x \geq b_{1} \vspace{2mm}\\
		& A_{2} x \leq b_{2} \vspace{2mm}\\
		& A_{3} x=b_{3} \vspace{2mm}\\
		& \boldsymbol{x} \geq 0 .
	\end{array}
\]

\textbf{Short answer}:

We transform the given problem into its standard form:

\[
	\begin{array}{rl}
		\operatorname{ minimize} & c^{\top} x \vspace{2mm} \\
		\text { subject to} & A_{1} x \geq b_{1} \vspace{2mm} \\
		& -A_{2} x \geq-b_{2} \vspace{2mm} \\
		& A_{3} x \geq b_{3} \vspace{2mm} \\
		& -A_{3} x \geq-b_{3} \vspace{2mm} \\
		& \boldsymbol{x} \geq 0 .
	\end{array}
\]

The dual problem is then:

\begin{equation*}
	\begin{aligned}
		& \text { maximize }\left[\begin{array}{llll}
			u^{\top} & v^{\top} & w^{\top} & \lambda^{\top}
		\end{array}\right]\left[\begin{array}{c}
			b_{1} \\
			-b_{2} \\
			b_{3} \\
			-b_{3}
		\end{array}\right] \\
		& \text { subject to }\left[\begin{array}{llll}
			u^{\top} & v^{\top} & w^{\top} & \lambda^{\top}
		\end{array}\right]\left[\begin{array}{c}
			A_{1} \\
			-A_{2} \\
			A_{3} \\
			-A_{3}
		\end{array}\right] \leq c^{\top}
	\end{aligned}
\end{equation*}

\begin{equation*}
	u, v, w, \lambda \geq 0 .
\end{equation*}


\medskip
\noindent
Example 3. Convert the following optimization problem into a linear programming problem and solve it;

\[
	\begin{array}{rl}
		\text { maximize} & -\left|x_{1}\right|-\left|x_{2}\right|-\left|x_{3}\right| \vspace{2mm}\\
		\text { subject to } & {\left[\begin{array}{ccc}
				1 & 1 & -1 \\
				0 & -1 & 0
			\end{array}\right]\left[\begin{array}{l}
				x_{1} \\
				x_{2} \\
				x_{3}
			\end{array}\right]=\left[\begin{array}{l}
				2 \\
				1
			\end{array}\right] .}
	\end{array}
\]

Construct its dual program and solve it.

\textbf{Short answer}:

We introduce two sets of non-negative variables: \(x_{i}^{+} \geq 0, x_{i}^{-} \geq 0\), \(i=1,2,3\). We can then represent the optimization problem in the form

\[
	\begin{array}{cl}
		\text { minimize} &
		\left(x_{1}^{+}+x_{1}^{-}\right)+\left(x_{2}^{+}+x_{2}^{-}\right)+\left(x_{3}^{+}+x_{3}^{-}\right) \vspace{2mm} \\
		\text {subject to} & {\left[\begin{array}{cccccc}
				1 & 1 & -1 & -1 & -1 & 1 \\
				0 & -1 & 0 & 0 & 1 & 0
			\end{array}\right]\left[\begin{array}{l}
				x_{1}^{+} \\
				x_{2}^{+} \\
				x_{3}^{+} \\
				x_{1}^{-} \\
				x_{2}^{-} \\
				x_{3}^{-}
			\end{array}\right]=\left[\begin{array}{l}
				2 \\
				1
			\end{array}\right]} \\
		& x_{i}^{+}, x_{i}^{-} \geq 0 .
	\end{array}
\]

We form the initial tableau,

\[
	\begin{array}{cccccc|c}
		1 & 1 & -1 & -1 & -1 & 1 & 2 \\
		0 & -1 & 0 & 0 & 1 & 0 & 1 \\
		\hline 1 & 1 & 1 & 1 & 1 & 1 & 0
	\end{array}
\]

There is no apparent basic feasible solution. We add the second row to the first one to obtain,

\[
	\begin{array}{cccccc|c}
		1 & 0 & -1 & -1 & 0 & 1 & 3 \\
		0 & -1 & 0 & 0 & 1 & 0 & 1 \\
		\hline 1 & 1 & 1 & 1 & 1 & 1 & 0
	\end{array}
\]

We next calculate the reduced cost coefficients,

\[
	\begin{array}{cccccc|c}
		1 & 0 & -1 & -1 & 0 & 1 & 3 \\
		0 & -1 & 0 & 0 & 1 & 0 & 1 \\
		\hline 0 & 2 & 2 & 2 & 0 & 0 & -4
	\end{array}
\]

We have zeros under the basic columns. The reduced cost coefficients are all non-negative. The optimal solution is,

\[
	\boldsymbol{x}^{*}=\left[\begin{array}{llllll}
		3 & 0 & 0 & 0 & 1 & 0
	\end{array}\right]^{\top} .
\]

The optimal solution to the original problem is \(\boldsymbol{x}^{*}=\left[\begin{array}{lll}3 & -1 & 0\end{array}\right]^{\top}\).

The dual of the above linear program is

\[
	\begin{array}{cl}
	\operatorname{maximize} &  2 \lambda_{1}+\lambda_{2} \\
	\text{ subject to} &  
		\left[\begin{array}{ll}\lambda_{1} & \lambda_{2}\end{array}\right]
		\left[\begin{array}{cccccc}1 & 1 & -1 & -1 & -1 & 1 \\ 0 & -1 & 0 & 0 & 1 & 0\end{array}\right] \leq
		\left[\begin{array}{llllll}1 & 1 & 1 & 1 & 1 & 1\end{array}\right].
	\end{array}
\]

The optimal solution to the dual is,

\[
	\begin{aligned}
		\boldsymbol{\lambda}^{* \top} & =\boldsymbol{c}_{\boldsymbol{B}}^{\top} \boldsymbol{B}^{-1} \\
		& =\left[\begin{array}{ll}
			1 & 1
		\end{array}\right]\left[\begin{array}{cc}
			1 & -1 \\
			0 & 1
		\end{array}\right]^{-1} \\
		& =\left[\begin{array}{ll}
			1 & 2
		\end{array}\right] .
	\end{aligned}
\]
