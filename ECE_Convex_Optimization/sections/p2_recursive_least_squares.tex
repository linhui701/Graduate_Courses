%------------------------------------------------------%
%------------------------------------------------------%
\section{Recursive Least Squares}
%------------------------------------------------------%
%------------------------------------------------------%

%------------------------------------------------------%
\subsection{Concept}
Lemma 2 Let \(\boldsymbol{A}\) be non-singular matrix. Let \(\boldsymbol{U}\) and \(\boldsymbol{V}\) be matrices such that \(\boldsymbol{I}+\boldsymbol{V} \boldsymbol{A}^{-1} \boldsymbol{U}\) is non-singular. Then, \(\boldsymbol{A}+\boldsymbol{U} \boldsymbol{V}\) is non-singular, and

\[
	(\boldsymbol{A}+\boldsymbol{U} \boldsymbol{V})^{-1}=\boldsymbol{A}^{-1}-\left(\boldsymbol{A}^{-1} \boldsymbol{U}\right)\left(\boldsymbol{I}+\boldsymbol{V} \boldsymbol{A}^{-1} \boldsymbol{U}\right)^{-1}\left(\boldsymbol{V} \boldsymbol{A}^{-1}\right) .
\]

We summarize by writing the recursive least squares algorithm using

\[
	\begin{aligned}
		\boldsymbol{P}_{k+1} & =\boldsymbol{P}_{k}-\boldsymbol{P}_{k} \boldsymbol{A}_{k+1}^{\top}\left(\boldsymbol{I}+\boldsymbol{A}_{k+1} \boldsymbol{P}_{k} \boldsymbol{A}_{k+1}^{\top}\right)^{-1} \boldsymbol{A}_{k+1} \boldsymbol{P}_{k}, \\
		\boldsymbol{x}^{(k+1)} & =\boldsymbol{x}^{(k)}+\boldsymbol{P}_{k+1} \boldsymbol{A}_{k+1}^{\top}\left(\boldsymbol{b}^{(k+1)}-\boldsymbol{A}_{k+1} x^{(k)}\right) .
	\end{aligned}
\]

In the special case where the new data at each step are such that \(\boldsymbol{A}_{k+1}\) is a matrix consisting of a single row, \(\boldsymbol{A}_{k+1}=\boldsymbol{a}_{k+1}^{\top}\), and \(b^{(k+1)}\) is a scalar, \(b^{(k+1)}=b_{k+1}\), we get

\[
	\begin{aligned}
		\boldsymbol{P}_{k+1} & =\boldsymbol{P}_{k}-\frac{\boldsymbol{P}_{k} \boldsymbol{a}_{k+1} \boldsymbol{a}_{k+1}^{\top} \boldsymbol{P}_{k}}{1+\boldsymbol{a}_{k+1}^{\top} \boldsymbol{P}_{k} \boldsymbol{a}_{k+1}}, \\
		\boldsymbol{x}^{(k+1)} & =\boldsymbol{x}^{(k)}+\boldsymbol{P}_{k+1} \boldsymbol{a}_{k+1}\left(b_{k+1}-\boldsymbol{a}_{k+1}^{\top} \boldsymbol{x}^{(k)}\right) .
	\end{aligned}
\]

%------------------------------------------------------%
\subsection{Examples}
Example 1: Let

\[
	\boldsymbol{A}=\left[\begin{array}{ll}
		0 & 1 \\
		1 & 1
	\end{array}\right], \boldsymbol{b}^{(0)}=\left[\begin{array}{l}
		1 \\
		1
	\end{array}\right]
\]

and

\[
	\boldsymbol{a}_{1}^{\top}=\left[\begin{array}{ll}
		1 & 0
	\end{array}\right], b^{(1)}=2 .
\]

Use the recursive least squares to solve the combined system of equations.

\textbf{Short answer}:

We first compute

\[
	\boldsymbol{P}_{0}=\left(\boldsymbol{A}^{\top} \boldsymbol{A}\right)^{-1}=\left(\left[\begin{array}{ll}
		0 & 1 \\
		1 & 1
	\end{array}\right]\left[\begin{array}{ll}
		0 & 1 \\
		1 & 1
	\end{array}\right]\right)^{-1}=\left[\begin{array}{ll}
		1 & 1 \\
		1 & 2
	\end{array}\right]^{-1}=\left[\begin{array}{cc}
		2 & -1 \\
		-1 & 1
	\end{array}\right] .
\]

Hence, in this example,

\[
	\boldsymbol{x}^{(0)}=\boldsymbol{P}_{0} \boldsymbol{A}_{0}^{\top} \boldsymbol{b}^{(0)}=\boldsymbol{A}_{0}^{-1} \boldsymbol{b}^{(0)}=\left[\begin{array}{ll}
		0 & 1 \\
		1 & 1
	\end{array}\right]\left[\begin{array}{l}
		1 \\
		1
	\end{array}\right]=\left[\begin{array}{l}
		0 \\
		1
	\end{array}\right] .
\]

We next compute

\[
	\begin{aligned}
		\boldsymbol{P}_{1} & =\boldsymbol{P}_{0}-\frac{\boldsymbol{P}_{0} \boldsymbol{a}_{1} \boldsymbol{a}_{1}^{\top} \boldsymbol{P}_{0}}{1+\boldsymbol{a}_{1}^{\top} \boldsymbol{P}_{0} \boldsymbol{a}_{1} } \\
		& =\left[\begin{array}{cc}
			2 & -1 \\
			-1 & 1
		\end{array}\right]-\frac{\left[\begin{array}{c}
				2 \\
				-1
			\end{array}\right]\left[\begin{array}{ll}
				2 & -1
			\end{array}\right]}{3} \\
		& =\left[\begin{array}{cc}
			\frac{2}{3} & -\frac{1}{3} \\
			-\frac{1}{3} & \frac{2}{3}
		\end{array}\right] .
	\end{aligned}
\]

Therefore,

\[
	\boldsymbol{x}^{(1)}=\boldsymbol{x}^{(0)}+\left(\boldsymbol{b}^{(0)}-\boldsymbol{a}_{1}^{\top} \boldsymbol{x}_{0}\right) \boldsymbol{P}_{1} \boldsymbol{a}_{1}=\left[\begin{array}{c}
		\frac{4}{3} \\
		\frac{1}{3}
	\end{array}\right] .
\]

Example 2: Find recursive least square solution for
\[
\begin{array}{rl}
	\boldsymbol{A}_0=\left[\begin{array}{ll}
		1 & 0 \\
		0 & 1 \\
		1 & 1
	\end{array}\right], 
	& \boldsymbol{b}^{(0)}=\left[\begin{array}{l}
		1 \\
		1 \\
		1
	\end{array}\right], \\
	& \\
	\boldsymbol{A}_1  =\boldsymbol{a}_1^{\top}=\left[\begin{array}{ll}
		2 & 1
	\end{array}\right], 
	& \boldsymbol{b}^{(1)}  =b_1=[3], \\
	\boldsymbol{A}_2  =\boldsymbol{a}_2^{\top}=\left[\begin{array}{ll}
		3 & 1
	\end{array}\right], 
	& \boldsymbol{b}^{(2)}  =b_2=[4] .
\end{array}
\]

\textbf{Short answer}:

Compute the vector \(\boldsymbol{x}^{(0)}\) minimizing \(\left\|\boldsymbol{A}_0 \boldsymbol{x}-\boldsymbol{b}^{(0)}\right\|^2\). 

We have
\[
\begin{aligned}
	\boldsymbol{P}_0 & =\left(\boldsymbol{A}_0^{\top} \boldsymbol{A}_0\right)^{-1}=\left[\begin{array}{cc}
		2 / 3 & -1 / 3 \\
		-1 / 3 & 2 / 3
	\end{array}\right], \\
	\boldsymbol{x}^{(0)} & =\boldsymbol{P}_0 \boldsymbol{A}_0^{\top} \boldsymbol{b}^{(0)}=\left[\begin{array}{c}
		2 / 3 \\
		2 / 3
	\end{array}\right]
\end{aligned}
\]

Then, use the RLS algorithm to find \(\boldsymbol{x}^{(2)}\) minimizing
\[
\left\|\left[\begin{array}{l}
	\boldsymbol{A}_0 \\
	\boldsymbol{A}_1 \\
	\boldsymbol{A}_2
\end{array}\right] \boldsymbol{x}-\left[\begin{array}{l}
	\boldsymbol{b}^{(0)} \\
	\boldsymbol{b}^{(1)} \\
	\boldsymbol{b}^{(2)}
\end{array}\right]\right\|^2 .
\]

\[
\begin{gathered}
	\boldsymbol{P}_1=\boldsymbol{P}_0-\frac{\boldsymbol{P}_0 \boldsymbol{a}_1 \boldsymbol{a}_1^{\top} \boldsymbol{P}_0}{1+\boldsymbol{a}_1^{\top} \boldsymbol{P}_0 \boldsymbol{a}_1}=\left[\begin{array}{cc}
		1 / 3 & -1 / 3 \\
		-1 / 3 & 2 / 3
	\end{array}\right], \\
	\boldsymbol{x}^{(1)}=\boldsymbol{x}^{(0)}+\boldsymbol{P}_1 \boldsymbol{a}_1\left(b_1-\boldsymbol{a}_1^{\top} \boldsymbol{x}^{(0)}\right)=\left[\begin{array}{c}
		1 \\
		2 / 3
	\end{array}\right], \\
	\boldsymbol{P}_2=\boldsymbol{P}_1-\frac{\boldsymbol{P}_1 \boldsymbol{a}_2 \boldsymbol{a}_2^{\top} \boldsymbol{P}_1}{1+\boldsymbol{a}_2^{\top} \boldsymbol{P}_1 \boldsymbol{a}_2}=\left[\begin{array}{cc}
		1 / 6 & -1 / 4 \\
		-1 / 4 & 5 / 8
	\end{array}\right], \\
	\boldsymbol{x}^{(2)}=\boldsymbol{x}^{(1)}+\boldsymbol{P}_2 \boldsymbol{a}_2\left(b_2-\boldsymbol{a}_2^{\top} \boldsymbol{x}^{(1)}\right)=\left[\begin{array}{c}
		13 / 12 \\
		5 / 8
	\end{array}\right] .
\end{gathered}
\]

We can easily check our solution by computing \(\boldsymbol{x}^{(2)}\) directly using the formula \(\boldsymbol{x}^{(2)}=\left(\boldsymbol{A}^{\top} \boldsymbol{A}\right)^{-1} \boldsymbol{A}^{\top} \boldsymbol{b}\), where
\[
\boldsymbol{A}=\left[\begin{array}{l}
	\boldsymbol{A}_0 \\
	\boldsymbol{A}_1 \\
	\boldsymbol{A}_2
\end{array}\right], \quad \boldsymbol{b}=\left[\begin{array}{l}
	\boldsymbol{b}^{(0)} \\
	\boldsymbol{b}^{(1)} \\
	\boldsymbol{b}^{(2)}
\end{array}\right].
\]


Example 3: 4. Given
\[
\boldsymbol{A}=\left[\begin{array}{lll}
	1 & 0 & 0 \\
	1 & 1 & 0 \\
	0 & 1 & 1
\end{array}\right], \quad \boldsymbol{u}=\left[\begin{array}{c}
	1 \\
	0 \\
	-2
\end{array}\right], \quad \text { and } \quad \boldsymbol{v}=\left[\begin{array}{lll}
	0 & 1 & 1
\end{array}\right]
\]

where

\[
\boldsymbol{A}^{-1}=\left[\begin{array}{ccc}
	1 & 0 & 0 \\
	-1 & 1 & 0 \\
	1 & -1 & 1
\end{array}\right] .
\]

Use the matrix inversion formula to find
\[
(\boldsymbol{A}-\boldsymbol{u} \boldsymbol{v})^{-1}.
\]

\textbf{Short answer}:

The matrix inversion formula for this problem takes the form
\[
(\boldsymbol{A}-\boldsymbol{u v})^{-1}=\boldsymbol{A}^{-1}+\frac{\boldsymbol{A}^{-1} \boldsymbol{u} \boldsymbol{v} \boldsymbol{A}^{-1}}{1-\boldsymbol{v} \boldsymbol{A}^{-1} \boldsymbol{u}}.
\]

Applying the above, we obtain
\[
\begin{aligned}
	(\boldsymbol{A}-\boldsymbol{u v})^{-1} & =\left[\begin{array}{ccc}
		1 & 0 & 0 \\
		-1 & 1 & 0 \\
		1 & -1 & 1
	\end{array}\right]+\frac{1}{3}\left[\begin{array}{c}
		1 \\
		-1 \\
		-1
	\end{array}\right]\left[\begin{array}{lll}
		0 & 0 & 1
	\end{array}\right] \\
	& =\left[\begin{array}{ccc}
		1 & 0 & 0 \\
		-1 & 1 & 0 \\
		1 & -1 & 1
	\end{array}\right]+\frac{1}{3}\left[\begin{array}{ccc}
		0 & 0 & 1 \\
		0 & 0 & -1 \\
		0 & 0 & -1
	\end{array}\right] \\
	& =\left[\begin{array}{ccc}
		1 & 0 & 1 / 3 \\
		-1 & 1 & -1 / 3 \\
		1 & -1 & 2 / 3
	\end{array}\right] .
\end{aligned}
\]
