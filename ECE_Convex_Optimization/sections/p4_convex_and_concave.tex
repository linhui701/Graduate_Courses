%------------------------------------------------------%
%------------------------------------------------------%
\section{Convex and concave}
%------------------------------------------------------%
%------------------------------------------------------%

%------------------------------------------------------%
\subsection{Practice problems}
\begin{enumerate}
	\item Find the range of values of the parameter \(\alpha\) for which the function,
\end{enumerate}

\[
	f\left(x_{1}, x_{2}, x_{3}\right)=2 x_{1} x_{3}-x_{1}^{2}-x_{2}^{2}-5 x_{3}^{2}-2 \alpha x_{1} x_{2}-4 x_{2} x_{3},
\]

is concave.

\textbf{Short answer}: A quadratic form is concave if and only if it is negative semi-definite. Equivalently, if and only if its negative is positive semi-definite. Therefore, we have

\[
	\alpha \in[-4 / 5,0] .
\]

(For more details please check Section 3 Problem 1 in this review material.)

\begin{enumerate}
	\setcounter{enumi}{1}
	\item Is the set
\end{enumerate}

\[
	S=\left\{\boldsymbol{x} \in \mathbb{R}^{2}:\left[\begin{array}{ll}
		1 & 2
	\end{array}\right] \boldsymbol{x} \leq 2\right\}
\]

convex or not?

\textbf{Short answer}: To show that the set \(S\) is convex, we need to demonstrate that if \(\boldsymbol{x}, \boldsymbol{y} \in S\), then so is the line segment

\[
	\alpha \boldsymbol{x}+(1-\alpha) \boldsymbol{y}, \alpha \in[0,1] .
\]

We have

\[
	\begin{aligned}
		{\left[\begin{array}{ll}
				1 & 2
			\end{array}\right](\alpha \boldsymbol{x}+(1-\alpha) \boldsymbol{y}) } & =\alpha\left[\begin{array}{ll}
			1 & 2
		\end{array}\right] \boldsymbol{x}+(1-\alpha)\left[\begin{array}{ll}
			1 & 2
		\end{array}\right] \boldsymbol{y} \\
		& \leq 2 \alpha+2(1-\alpha) \\
		& =2
	\end{aligned}
\]

that is, if \(\boldsymbol{x}, \boldsymbol{y} \in S\), then \(\alpha \boldsymbol{x}+(1-\alpha) \boldsymbol{y} \in S\) for \(\alpha \in[0,1]\). Therefore, the set \(S\) is convex.

\begin{enumerate}
	\setcounter{enumi}{2}
	\item Given a monotone non-decreasing function \(g\) of single variable, that is, \(g\left(r_{1}\right) \leq g\left(r_{2}\right)\) for \(r_{1}<r_{2}\). The function \(g\) is also convex. Let \(f\) be a convex function on a convex set \(\Omega \subseteq \mathbb{R}^{n}\). Show that the composite function \(g(f(\boldsymbol{x}))\) is convex on \(\Omega\).
\end{enumerate}

\textbf{Short answer}: Let \(\boldsymbol{x}, \boldsymbol{y} \in \Omega\). Since \(\Omega\) is convex, we have \(\alpha \boldsymbol{x}+(1-\alpha) \boldsymbol{y} \in \Omega\), for \(\alpha \in[0,1]\). Since \(f\) is convex, we have \(f(\alpha \boldsymbol{x}+(1-\alpha) \boldsymbol{y}) \leq \alpha f(\boldsymbol{x})+(1-\alpha) f(\boldsymbol{y})\). Now let \(r_{1}=f(\alpha \boldsymbol{x}+(1-\alpha) \boldsymbol{y})\) and \(r_{2}=\alpha f(\boldsymbol{x})+(1-\alpha) f(\boldsymbol{y})\). Then \(r_{1} \leq r_{2}\) implies \(g\left(r_{1}\right) \leq g\left(r_{2}\right)\). Since \(r\) is a function of \(f, g(r)\) is convex, and \(\boldsymbol{x} \in \Omega\), we have \(g(f(\boldsymbol{x}))\) to be convex on \(\Omega\).
