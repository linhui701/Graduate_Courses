%------------------------------------------------------%
%------------------------------------------------------%
\section{Pseudoinverse}
%------------------------------------------------------%
%------------------------------------------------------%

\subsection{Concept}
\begin{itemize} 
	\item[\(\spadesuit\)] \(\mathscr{THEOREM}\) Given \(\boldsymbol{A} \in \mathbb{R}^{m \times n}\), a matrix \(\boldsymbol{A}^{\dagger} \in \mathbb{R}^{n \times m}\) is called a pesudoinverse of the matrix \(\boldsymbol{A}\) if
	\[
		\boldsymbol{A} \boldsymbol{A}^{\dagger} \boldsymbol{A}=\boldsymbol{A},
	\]

	and there exist matrices \(\boldsymbol{U} \in \mathbb{R}^{n \times n}, \boldsymbol{V} \in \mathbb{R}^{m \times m}\) such that

	\[
		\boldsymbol{A}^{\dagger}=\boldsymbol{U} \boldsymbol{A}^{\top}, \text { and } \boldsymbol{A}^{\dagger}=\boldsymbol{A}^{\top} \boldsymbol{V} .
	\]

\end{itemize}

For the case in which a matrix \(\boldsymbol{A} \in \mathbb{R}^{m \times n}\) with \(m \geq n\) and \(\operatorname{rank}(\boldsymbol{A})=n\), We have the \underline{left pseudoinverse} of \(\boldsymbol{A}\) :

\[
	\boldsymbol{A}^{\dagger}=\left(\boldsymbol{A}^{\top} \boldsymbol{A}\right)^{-1} \boldsymbol{A}^{\top} .
\]

For the case in which a matrix \(\boldsymbol{A} \in \mathbb{R}^{m \times n}\) with \(m \leq n\) and \(\operatorname{rank}(\boldsymbol{A})=m\), We have the \underline{right pseudoinverse} of \(\boldsymbol{A}\) :

\[
	\boldsymbol{A}^{\dagger}=\boldsymbol{A}^{\top}\left(\boldsymbol{A} \boldsymbol{A}^{\top}\right)^{-1} .
\]

\begin{itemize}
	\item[\(\spadesuit\)] \(\mathscr{THEOREM}\) Let \(\boldsymbol{A} \in \mathbb{R}^{m \times n}\). If a pseudoinverse \(\boldsymbol{A}^{\dagger}\) of \(\boldsymbol{A}\) exists, then it is unique.

	\item[\(\spadesuit\)] \(\mathscr{THEOREM}\) Let \(\boldsymbol{A} \in \mathbb{R}^{m \times n}\) have a full-rank factorization \(\boldsymbol{A}=\boldsymbol{B} \boldsymbol{C}\), with \(\operatorname{rank}(\boldsymbol{A})=\operatorname{rank}(\boldsymbol{B})=\operatorname{rank}(\boldsymbol{C})=r, \boldsymbol{B} \in \mathbb{R}^{m \times r}, \boldsymbol{C} \in \mathbb{R}^{r \times n}\). Then,

	\[
		\boldsymbol{A}^{\dagger}=\boldsymbol{C}^{\dagger} \boldsymbol{B}^{\dagger} .
	\]
\end{itemize}


%------------------------------------------------------%
\subsection{Example}
Example 1: Simplify
\[
	\boldsymbol{A}^{\top} \boldsymbol{A} \boldsymbol{A}^{\dagger}
\]


\textbf{Short answer}:

Let \(\boldsymbol{A} \in \mathbb{R}^{m \times n}\).

Case 1:

If \(\operatorname{rank}(\boldsymbol{A})=n>m\), we have

\[
	\boldsymbol{A}^{\dagger}=\left(\boldsymbol{A}^{\top} \boldsymbol{A}\right)^{-1} \boldsymbol{A}^{\top} .
\]

Then

\[
	\boldsymbol{A}^{\top} \boldsymbol{A} \boldsymbol{A}^{\dagger}=\boldsymbol{A}^{\top} \boldsymbol{A}\left(\boldsymbol{A}^{\top} \boldsymbol{A}\right)^{-1} \boldsymbol{A}^{\top}=\boldsymbol{A}^{\top} .
\]

Case 2:

If \(\operatorname{rank}(\boldsymbol{A})=m>n\), we have

\[
	\boldsymbol{A}^{\dagger}=\boldsymbol{A}^{\top}\left(\boldsymbol{A} \boldsymbol{A}^{\top}\right)^{-1} .
\]

Then

\[
	\boldsymbol{A}^{\top} \boldsymbol{A} \boldsymbol{A}^{\dagger}=\boldsymbol{A}^{\top} \boldsymbol{A} \boldsymbol{A}^{\top}\left(\boldsymbol{A} \boldsymbol{A}^{\top}\right)^{-1}=\boldsymbol{A}^{\top}
\]

\textbf{Case 3:}
If \(\operatorname{rank}(\boldsymbol{A})=r\) and \(r<\min \{m, n\}\), we have

\[
	\boldsymbol{A}=\boldsymbol{B} \boldsymbol{C},
\]

where \(\boldsymbol{B} \in \mathbb{R}^{m \times r}\), and \(\boldsymbol{C} \in \mathbb{R}^{r \times n}\). Then

\[
	\boldsymbol{A}^{\dagger}=\boldsymbol{C}^{\dagger} \boldsymbol{B}^{\dagger}
\]

where

\[
	\boldsymbol{B}^{\dagger}=\left(\boldsymbol{B}^{\top} \boldsymbol{B}\right)^{-1} \boldsymbol{B}^{\top}
\]

and

\[
	\boldsymbol{C}^{\dagger}=\boldsymbol{C}^{\top}\left(\boldsymbol{C} \boldsymbol{C}^{\top}\right)^{-1}
\]

Hence,

\[
	\begin{aligned}
		\boldsymbol{A}^{\top} \boldsymbol{A} \boldsymbol{A}^{\dagger} & =\boldsymbol{C}^{\top} \boldsymbol{B}^{\top} \boldsymbol{B} \boldsymbol{C} \boldsymbol{C}^{\dagger} \boldsymbol{B}^{\dagger} \\
		& =\boldsymbol{C}^{\top} \boldsymbol{B}^{\top} \boldsymbol{B} \boldsymbol{C} \boldsymbol{C}^{\top}\left(\boldsymbol{C} \boldsymbol{C}^{\top}\right)^{-1}\left(\boldsymbol{B}^{\top} \boldsymbol{B}\right)^{-1} \boldsymbol{B}^{\top} \\
		& =\boldsymbol{C}^{\top} \boldsymbol{B}^{\top}=A^{\top} .
	\end{aligned}
\]

Therefore, in conclusion, \(\boldsymbol{A}^{\top} \boldsymbol{A} \boldsymbol{A}^{\dagger}=\boldsymbol{A}^{\top}\).
