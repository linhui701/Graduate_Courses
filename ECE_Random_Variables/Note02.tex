\documentclass[12pt,thmsa]{article}

\usepackage{amsmath}
\usepackage{amsfonts} % for \mathbb
\usepackage{algorithm}% http://ctan.org/pkg/algorithms
\usepackage{algpseudocode}% http://ctan.org/pkg/algorithmicx
\usepackage{graphicx} % Allows including images
\usepackage{booktabs} % Allows the use of \toprule, \midrule and \bottomrule in tables
\usepackage{cancel}
\usepackage{color}    % color
\usepackage{geometry}
\usepackage{amssymb}  % \varnothing
\usepackage{enumitem} % enumerate
\usepackage{textcase} % \MakeTextUppercase
\usepackage{float}    % Force figure placement in text with [H]
\geometry{
	a4paper,
	total={170mm,257mm},
	left=20mm,
	right=20mm,
	top=20mm,
}

% Linhui added for newly defined color
\definecolor{forestgreen}{RGB}{34,139,34}


% Linhui added for Expectation and Variance
\newcommand{\Exp}{{\mathbb E}\! }
\newcommand{\Var}{\mbox{Var}\! }

% Linhui added for rename the command for empty set.
\let\oldemptyset\emptyset
\let\emptyset\varnothing


%------------------------------------------------------%
\makeatletter
\def\maketitle{%
	\par
	\hrule height 1.5pt\vspace{1ex}
	\par\noindent
	
	\begin{minipage}{0.5\textwidth}
		\scshape
		Purdue University $\cdot$ ece 60000 \\[1ex]
		Random Variables and Signals \\
		Prof. Bell, Prof. Salama
	\end{minipage}
	\begin{minipage}{0.45\textwidth}
		\raggedleft
		\MakeTextUppercase{{\@title}}\\[0.3ex] % 0.2ex height space between two line
		\textit{\@author}\\[0.2ex]
		\textit{September 30, 2021}
	\end{minipage}
	\par\vspace{1ex}
	\hrule height 1.5pt\vspace{1ex}
	\par
}
\makeatother

\author{Linhui Xie}
\title{Lecture Note 2}
%------------------------------------------------------%

\begin{document}
\maketitle

\setcounter{section}{1}
\section{Lecture 2\medskip}

\setcounter{section}{2}
%------------------------------------------------------%
\subsection{Fundamental Set Operations}  % 2.1

[Recall] There are 3 fundamental set operations:
\begin{itemize}
	\item Union: \(
	A \cup B =\{ {\color{forestgreen}{\omega \in \Omega}}: \omega \in A {\color{blue}{\text { or }}} \omega \in B\}.
	\)
	\item Intersection: \(
	A \cap B=\{ {\color{forestgreen}{\omega \in \Omega}}: \omega \in A {\color{blue}{\text { and }}} \omega \in B\}.
	\)
	\item Complement: \(
	\overline{A}=A^c=\{ {\color{forestgreen}{\omega \in \Omega}}: \omega \notin  A\}.
	\)
\end{itemize}

\noindent
There are 2 other ``set difference operations'' that are sometimes used:
\begin{itemize}
	\item \(A - B = \{{\color{forestgreen}{\omega}} \in A: \omega \notin B\} \)
	\item \(B - A = \{{\color{forestgreen}{\omega}} \in B: \omega \notin A\} \)
\end{itemize}

%------------------------------------------------------%
\subsection{Index sets \(\mathcal{I}\)} % 2.2
[Recall]
\begin{itemize}
	\item   Indexed collections of sets \( \{A_{i} ; i \in \mathcal{I}\} \), where \(\mathcal{I} \) is an index set.
	\item   So \( \{A_{i} ; i \in \mathcal{I}\} \) is a ``set of sets", or a \underline{family of sets}, or a \underline{collection of sets}.
	\item  There is one and only one set \( A_{i} \), for each \( i \in \mathcal{I} \).

	Typical Index sets:
	\[\begin{array}{l} 
		\mathbb{N}=\{1,2,3, \ldots\}=\text { natural numbers }, \\ 
		\mathbb{Z}=\{\ldots-1,0,1,2, \ldots\}=\text { integers }, \\ 
		\mathbb{Z}_{+}=\{0,1,2, \ldots\}=\text { non-negative integers }, \\
		\mathbb{R}=(-\infty, \infty), \\
		\mathbb{I}_{n}=\{0,1,2, \ldots, n-1\}.
	\end{array}\]
\end{itemize}

%------------------------------------------------------%
\subsection{Cardinality of Sets} % 2.3

\begin{itemize}
	\item A set is \underline{finite} if it has a finite number of elements.
	
	(i.e., its elements can be put in one-to-one correspondence with the numbers $1, 2, \ldots, n$ for some natural number $n$.)
	
	\item A set is \underline{infinite} if it is not finite.
\end{itemize}

%------------------------------------------------------%
\subsection{Countable and Uncountable Sets} % 2.4
Infinite sets come in two varieties: \textit{countable} and \textit{uncountable}.

\begin{itemize}
	\item An infinite set is \underline{countable} if its elements can be put in one-to-one correspondence with the natural (counting) numbers \(\mathbb{N} = \{1, 2, 3, \ldots\}\).
	
	\item An infinite set is \underline{uncountable} if it is not countable.
	
	\item The following are examples of uncountable sets:
	\begin{itemize}
		\item \(\mathbb{R} = (-\infty, \infty)\)
		\item \([0, 1]\) and \((0, 1)\)
		\item \([a, b]\), \([a, b)\), \((a, b]\), \((a, b)\) for \(a, b \in \mathbb{R}\) such that \(a < b\).
	\end{itemize}
\end{itemize}

%------------------------------------------------------%
\subsection{Collectively Exhaustive and Disjoint Families} % 2.5

\begin{itemize}
	\item Given an indexed family of sets $\{A_i; i \in I\}$, the union of the sets in the family is
	\[ 
	{\color{blue}{\bigcup_{i \in I}}} A_{i} \triangleq\left\{ {\color{forestgreen}{\omega \in \Omega}}: \omega \in A_{i} \text { for } {\color{blue}{ \text{at least}}} \text{ one } i \in \mathcal{I} \right\}.
	\]
	\item The intersection of the sets in the family is
	\[
	{\color{blue}{\bigcap_{i \in I}}} A_{i} \triangleq\left\{ {\color{forestgreen}{\omega \in \Omega}}: \omega \in A_{i} \text { for } {\color{blue}{ \text{all } }} i \in \mathcal{I} \right\}.
	\]

	\item If $G \subseteq \Omega$ and $\{A_i; i \in \mathcal{I}\}$ is a family of sets, then if
	\[
	\bigcup_{i \in \mathcal{I}} A_i = G
	\]
	we say that $\{A_i; i \in \mathcal{I}\}$ is \underline{collectively exhaustive} over $G$.
	
	\item A family of sets $\{A_i; i \in \mathcal{I}\}$ is \underline{disjoint} if
	\[
	A_i \cap A_j = \emptyset, \forall i, j \in \mathcal{I} \text{ such that } i \neq j.
	\]
\end{itemize}

%------------------------------------------------------%
\subsection{Partitioned Sets} % 2.6

\begin{itemize}	
	\item A family of sets \(\{A_i; i \in \mathcal{I}\}\) is a \underline{partition} of \(\Omega\) if it is disjoint and collectively exhaustive over \(\Omega\), meaning:
	\begin{itemize}
		\item \textit{Disjointness:} \(A_i \cap A_j = \emptyset\) for all \(i \neq j\).
		\item \textit{Collectively exhaustive:} \(\bigcup_{i \in I} A_i =\Omega \), every element of \(\Omega\) is in at least one \(A_i\).
	\end{itemize}
	
	\item Let \(\{A_i; i \in \mathcal{I}\}\) be a partition of \(\Omega\). Define \(B_i \triangleq A_i \cap G\) for some \(G \subseteq \Omega\) and for all \(i \in \mathcal{I}\). Show that, \(\{B_i; i \in \mathcal{I}\}\) is a partition of \(G\).
	
	Need to prove,
	\begin{enumerate}
		\item The family of sets  \(\{B_i; i \in \mathcal{I}\}\) are disjoint.
		
		Assume \(B_i\) and \(B_j\) are not disjoint for some \(i \neq j\), meaning there exists an element \(x\) such that \(x \in B_i\) and \(x \in B_j\). By definition, \(x \in A_i \cap G\) and \(x \in A_j \cap G\), implying \(x \in A_i\) and \(x \in A_j\), which contradicts the assumption that \(\{A_i\}\) are disjoint. Therefore,\(\{B_i; i \in \mathcal{I}\}\) is disjoint.
		
		\item The union of all \(\{B_i; i \in \mathcal{I}\}\) is \(G\).
		
		Since \(B_i = A_i \cap G\), then \(\bigcup_{i \in \mathcal{I}} B_i = \bigcup_{i \in \mathcal{I}} (A_i \cap G)\). Given \(\bigcup_{i \in \mathcal{I}} A_i = \Omega\) and \(G \subseteq \Omega\), it follows that \(\bigcup_{i \in \mathcal{I}} (A_i \cap G) = G = \bigcup_{i \in \mathcal{I}} B_i\).
		
		Combine 1 and 2, \(\{B_i; i \in \mathcal{I}\}\) forms a partition of \(G\).
	\end{enumerate}

\end{itemize}


%------------------------------------------------------%
\subsection{Probability Spaces \((\Omega, \mathcal{F}, P)\) } % 2.7

A \underline{probability space} \((\Omega, \mathcal{F}, P)\) is a triple consisting of:
\begin{itemize}
	\item A sample space \(\Omega\) (or hand written notation \(\mathcal{S}\) in this course).
	\item A collection of events (subsets of \(\Omega\)) \(\mathcal{F}(\Omega)\).
	\item The probabilities \(P(A)\) for each \(A \in \mathcal{F}( \Omega )\),
			\[ P:  \mathcal{F}(\Omega) \rightarrow [ 0,1 ].
			\]
%	The probabilities \(P(A)\) satisfying the axioms of probability:
%	\begin{enumerate}
%		\item \(P(A) \geq 0\) for all \(A \in \mathcal{F}\).
%		\item \(P(\Omega) = 1\).
%		\item If \(A_1, A_2, \ldots\) are disjoint events in \(\mathcal{F}\), then \(P(\bigcup_{i} A_i) = \sum_{i} P(A_i)\).
%	\end{enumerate}

	For example, consider the random experiment of flipping a fair coin. The probability space for this experiment:
	\begin{itemize}
		\item \(\Omega = \{\text{``Heads"}, \text{``Tails"}\}\),
		\item \(\mathcal{F} = \{\emptyset, \{\text{``Heads"}\}, \{\text{``Tails"}\}, \Omega\}\),
		\item \(P(\{\text{``Heads"}\}) = 0.5\), \(P(\{\text{``Tails"}\}) = 0.5\), \(P(\Omega) = 1\), and \(P(\emptyset) = 0\).
	\end{itemize}
\end{itemize}

%------------------------------------------------------%
\subsection{The Sample Space \(\Omega\)} % 2.8

\begin{itemize}
	\item The \underline{sample space} \(\Omega\) of a random experiment is a non-empty set of possible outcomes of the random experiment.

	\fbox{\parbox{0.9\textwidth}{
			One and only one outcome from the sample space \(\Omega\) occurs when we perform a random experiment.
	}}
\end{itemize}

%------------------------------------------------------%
\subsection{The Event Space \( \mathcal{F}(\Omega) \)} % 2.9
The \underline{event space} \( \mathcal{F}(\Omega) \) is a non-empty collection of subsets of \(\Omega\) satisfying the following closure properties:

\begin{enumerate}
	\item If \( A \in \mathcal{F}(\Omega) \), then \( \overline{A} \in \mathcal{F}(\Omega) \).
	\item For any finite \( n \), if \( A_i \in \mathcal{F}(\Omega) \) for \( i = 1, 2, \ldots, n \), then \( \bigcup_{i=1}^{n} A_i \in \mathcal{F}(\Omega) \).
	\item If \( A_i \in \mathcal{F}(\Omega) \) for \( i = 1, 2, 3, \ldots \), then \( \bigcup_{i=1}^{\infty} A_i \in \mathcal{F}(\Omega) \).
\end{enumerate}

A set of subsets of \(\Omega\) satisfying these three properties is called a \underline{\( \sigma \)-field} (\(\mathcal{F}_{\sigma}\)). 

\textit{Note: If only 1 and 2 hold, we have a \underline{field} of sets.}

%------------------------------------------------------%
\subsection{Intersections in the Event Space \( \mathcal{F}(\Omega) \)} %2.10
\begin{itemize}
	\item Suppose \( A, B \in \mathcal{F}(\Omega) \), is \( A \cap B \in \mathcal{F}(\Omega) \)?
	
	Short answer, by the closure property 1, closed under complements:
	\[ \overline{A}, \overline{B} \in \mathcal{F}(\Omega). \]
	
	Using the closure property 2, closed under unions:
	\[ \overline{A} \cup \overline{B} \in \mathcal{F}(\Omega). \]
	
	Applying De Morgan's laws:
	\[ A \cap B  = \overline{\overline{A \cap B}} = \overline{\overline{A} \cup \overline{B}}. \]
	
	Since the event space is closed under complements, it follows that:
	\[ \overline{\overline{A} \cup \overline{B}} \in \mathcal{F}(\Omega), \]
	thus proving \( A \cap B \in \mathcal{F}(\Omega) \).
	
	\item It follows from the closure properties that \( \emptyset, \Omega \in \mathcal{F}(\Omega) \).
	
	Proof:
	
	Suppose \( A \in \mathcal{F}(\Omega) \) (non-empty), by closure property 1, closed under complements,
	\[ \overline{A} \in \mathcal{F}(\Omega). \]
	
	Furthermore, by closure property 2, closed under unions,
	\[ \Omega = A \cup \overline{A} \in \mathcal{F}(\Omega). \]
	
	By closure property 1 again, closed under complements,
	\[ \Rightarrow \emptyset = \overline{\Omega} \in \mathcal{F}(\Omega) \] 
	
	Hence, \( \emptyset, \Omega \in \mathcal{F}(\Omega) \).
\end{itemize}


%------------------------------------------------------%
\subsection{Examples}
1. Show that the power set of any set \(\Omega,\) is a \(\sigma\)-field (\(\mathcal{F}_{\sigma}\)).

Note: the power set of \(\Omega\), is the set contains all subsets of \(\Omega\) including \(\Omega\) and \(\emptyset\).

\noindent
Short answer,

Let \(\rho(\Omega)\) be the power set of \(\Omega\), then
\begin{enumerate}[label=\roman*)] % i), ii), iii), ...
	\item \(\Omega \in \rho(\Omega)\), it's non-empty. \\
	(By definition, the power set contains all subsets of \(\Omega\) including \(\Omega\).)
	
	\item \textbf{For any} \(A \in \rho(\Omega), \quad \overline{A} \in \rho(\Omega)\). \\
	(as \(\rho(\Omega)\) contains all subsets of \(\Omega\).)
	\item \textbf{For any} \(A_{1}, A_{2}, A_{3}, \cdots \in \rho(\Omega)\), \(\begin{aligned}\bigcup_{n=1}^{\infty} A_{n}\end{aligned}\) is also a subset of \(\Omega\). Hence, \(\begin{aligned}\bigcup_{n=1}^{\infty} A_{n}\end{aligned} \in \rho(\Omega)\).
\end{enumerate}

Thus, \(\rho(\Omega)\) is a \(\sigma\)-field.

\medskip

%------------------------------------------------------%
%------------------------------------------------------%

\noindent
2. Let \(A\) and \(B\) be a two \(\sigma\)-fields of subsets of \(\Omega\). Show that \(A \cap B\) is also a \(\sigma\)-fields of
subsets of \(\Omega\).

\noindent
Short answer,

Need to verify that the intersection of \(A\) and \(B\) satisfies the three properties:
\begin{enumerate}[label=\roman*)] % i), ii), iii)
	\item The sample space \(\Omega\) is in the intersection.
	\item The intersection is closed under complements.
	\item The intersection is closed under countable unions.
\end{enumerate}

Given A is \(\sigma\)-field of \textbf{subsets of} \(\Omega\), and B is \(\sigma\)-field of \textbf{subsets of} \(\Omega\).

Let \(D=A \cap B\), then,
\begin{enumerate}[label=\roman*)] % i), ii), iii), ...
	\item A is \(\sigma\)-field of \textbf{subsets of} \(\Omega\), and B is \(\sigma\)-field of \textbf{subsets of} \(\Omega\). \\
	\(\Rightarrow \Omega \in A\) and \(\Omega \in B\), \\
	\(\Rightarrow \Omega \in A \cap B,\) \\
	\(\Rightarrow \Omega \in D.\)
	\item \textbf{For any} \(G \in D\), \(G\) is also in \(A\) (\(G \in A\)) \& in \(B\) (\(G \in B\)). A is \(\sigma\)-field, and B is \(\sigma\)-field. Thus, \(\overline{G} \in A\) and \(\overline{G} \in B,\) \\
	\(\Rightarrow \overline{G} \in A \cap B,\) \\
	\(\Rightarrow \overline{G} \in D.\)
	\item \textbf{For any} \(G_{1}, G_{2}, G_{3}, \cdots \in D\), \(G_{1}, G_{2}, \cdots\) in \(A\) and \(G_{1}, G_{2}, \ldots\) in \(B\). A is \(\sigma\)-field, and B is \(\sigma\)-field. \\
	\(\Rightarrow \begin{aligned}\bigcup_{n=1}^{\infty} G_{n} \in A\end{aligned}\) and \(\begin{aligned}\bigcup_{n=1}^{\infty} G_{n} \in B\end{aligned},\) \\
	\(\Rightarrow  \begin{aligned}\bigcup_{n=1}^{\infty} G_{n} \in A \cap B\end{aligned},\) \\
	\(\Rightarrow \begin{aligned}\bigcup_{n=1}^{\infty} G_{n} \in D\end{aligned}.\)
\end{enumerate}

Thus, \(A \cap B\) is also a \(\sigma\)-field of \textbf{subsets of} \(\Omega\).

\medskip

%------------------------------------------------------%
%------------------------------------------------------%

\noindent
3. Let \(B\) be a \(\sigma\)-field of subsets of the real line \(\mathbb{R}\). \(B\) is generated by intervals of the form (a,b) where \(a<b, -\infty<a<\infty, -\infty<b<\infty\), that is, \(B=\sigma\)(all finite open intervals). Show that
\begin{enumerate}[label=\alph*)] % a), b), c), ...
	\item \(B\) contains all sets of the form \((-\infty, a)\)
	
	\item \(B\) contains all sets of the form \((-\infty, a]\)
\end{enumerate}

\noindent
Short answer,

a)

Because \(B=\sigma\)(all finite open intervals), \textbf{for any} positive integer \(n\), \((a-n, a)\) is a subset of \(B\).
\[
(a-n, a) \in B
\]

Because \(B\) is \(\sigma\)-field of real line, for \(\sigma\)-field property iii)
\[
\begin{aligned}
	& \bigcup_{n=1}^{\infty}(a-n, a)& &\in B \\
	\Rightarrow &\quad (-\infty, a)& &\in B .
\end{aligned}
\]

b)

Given \(B=\sigma\)(all finite open intervals),

we set \(G_{n}\), for any positive integer \(n\),the open interval \(\left(-\infty, a+\frac{1}{n}\right)\) is a subset of \(B\).

we set \(H_{m}\), for any positive integer \(m\),the open interval \(\left(-\infty, a+\frac{1}{m}\right)\) is a subset of \(B\).

\(\overline{G}, \overline{H} \in B\) [ by ii) ], \(\overline{G} \cup \overline{H} \in B\) [ by iii) ], \( \overline{\overline{G} \cup \overline{H}} \in B\) [ by ii) ].

\[\Rightarrow \quad G \cap H \in B, \quad \text{for countable intersections.} \]

\[
\bigcap_{n=1}^{\infty} G_{n}
=\overline{\overline{\bigcap_{n=1}^{\infty} G_{n}}}
=\overline{\bigcup_{n=1}^{\infty} \overline{{G}}_{n}} \in B
\]

\[
\text { that is, } \sum_{n=1}^{\infty} \left(-\infty, a+\frac{1}{n}\right) \in B
\]

\[
\Rightarrow \quad(-\infty, a] \in B
\]

\medskip

%------------------------------------------------------%
%------------------------------------------------------%


\noindent
4. A class \(A\) of subsets of \(\Omega\) is said to be monotone if the limit of any monotone increasing/decreasing sequence of sets in \(A\) is also in \(A\). Show that a \(\sigma\)-field is monotone.

\noindent
Short answer,

Let \(A\) be a monotone \textbf{class} of \textbf{subsets of} \(\Omega\). 

i.e. for any \textbf{M}onotone \textbf{I}ncreasing \textbf{S}ubsets (MIS) of events,
\[
B_{n} \uparrow B, \quad B \in A.
\]

Similarly, for any \textbf{M}onotone \textbf{D}ecreasing \textbf{S}ubsets (MDS) of events,
\[
D_{n} \downarrow D, \quad D \in A.
\]

\(\Longrightarrow\)

Let \(\mathcal{F}\) be any \(\sigma\)-field of \textbf{subsets of} \(\Omega\). Suppose \(B_{1} \subset B_{2} \subset \cdots\) is an MIS sequence of events in \(\mathcal{F}\). Becanse \(\mathcal{F}\) is a \(\sigma\)-field, then
\[
\bigcup_{n=1}^{\infty} B_{n} \in \mathcal{F} \quad \Rightarrow \quad \lim _{n \rightarrow \infty} B_{n} \in \mathcal{F}.
\]

Similarly, for any MDS sequence of events in \(\mathcal{F}\), \(D_{1} \supset D_{2} \supset \dots\)
\[
\bigcap_{n=1}^{\infty} D_{n} \in \mathcal{F} \quad \Rightarrow \quad \lim _{n \rightarrow \infty} D_{n} \in \mathcal{F}.
\]

Because \(\mathcal{F}\) is a \(\sigma\)-field,
\[\Rightarrow \mathcal{F} \text{ is monotone.}\]

\(\Longleftarrow\) conversely,

Suppose \(G\) is a \textbf{field} of \textbf{subsets of} \(\Omega\) that
is monotone, we can show that \(G\) is a \(\sigma\)-field.

\begin{enumerate}[label=\roman*)] % i), ii), iii), ...
	\item Becanse \(G\) is a \textbf{field}, \\
	\(\Rightarrow \Omega \in G\).
	
	\item Becanse \(G\) is a \textbf{field}, \textbf{for any} \(A \in G, \quad \overline{A} \in G\)
	
	\item Now consider any arbitary sequence of events \(A_{1}, A_{2}, A_{3}, \cdots\), We need to show that \(\begin{aligned}\bigcup_{n=1}^{\infty} A_{n} \in G\end{aligned}\). Define a new sequence of events,
	\[
	\begin{aligned}
		B_{1} &=A_{1} \quad &\in G \\
		B_{2} &=A_{1} \cup A_{2} \quad &\in G \\
		\vdots & & \\
		B_{n} &= \bigcup_{j=1}^{n} A_{j} \quad &\in G \\
	\end{aligned}
	\]
\end{enumerate}

Any finite union is in the \textbf{field} \(G\)(field Definition). We know \(B_{1} \subset B_{2} \subset B_{3} \subset \cdots\), \(\begin{aligned}\bigcup_{j=1}^{n} B_{j} \in G \end{aligned}\).
\[
\text { i.e. this is an \textbf{M}onotone \textbf{I}ncreasing \textbf{S}ubsets (MIS) of events.}
\]

Because \(G\) is monotone, by the definition, the limit of any monotone increasing/decreasing sequence of sets in \(G\) is also in \(G\). That is, \[\lim_{n \rightarrow \infty} \bigcup_{j=1}^{n} B_{j} \in G.\]

The monotone limits of sequence is still in \(G\). Give that \(\begin{aligned}\lim_{n \rightarrow \infty} \bigcup_{j=1}^{n} B_{j}=\bigcup_{n=1}^{\infty} A_{n}\end{aligned}\),
\[
\Rightarrow \quad \bigcup_{n=1}^{\infty} A_{n} \in G.
\]

Combine (i), (ii) and (iii), \(G\) is a \(\sigma\)-field.


\medskip

%------------------------------------------------------%
%------------------------------------------------------%

\noindent
{\bf 5.} Suppose \((\Omega, \mathcal{F}, P)\) is a probability space with \(B \in \mathcal{F}\) such that \(P(B)>0 \). Define a new set function \(Q: \mathcal{F} \rightarrow[0,1]\) by \(Q(A)=P(A \mid B)\) for any \(A \in \mathcal{F} \). Show that
\begin{enumerate}[label=\alph*)] % a), b), c), ...
	\item \((\Omega, \mathcal{F}, Q)\) is a probability space.
	
	\item If \(C \in \mathcal{F}\) such that \(Q(C)>0\), show that \(Q(A \mid C)=P(A \mid B \cap C)\).
\end{enumerate}

For any \(A \in \mathcal{F}\),
\[
Q(A)=P(A \mid B)=\frac{P(A \cap B)}{P(A)}
\]

\noindent
Short answer,

a)

To show \((\Omega, \mathcal{F}, Q)\) is probability space, we need to show,

\begin{enumerate}% 1., 2., 3., ...
	\item \(Q(A) \geqslant 0\). \\
	We know that \(\begin{aligned} Q(A)=\frac{P(A \cap B)}{P(A)} = \frac{P(A \cap B) \geqslant 0}{P(A) > 0} \end{aligned}\). Thus, \( Q(A) \geqslant 0\)
	
	
	\item \(Q(\Omega)=1\). \\
	\[Q(\Omega)=P(\Omega \mid B)=\frac{P(\Omega \cap B)}{P(B)}=\frac{P(B)}{P(B)}=1.\]
	
	\item For any disjoint sets \( \begin{aligned} A_{1} A_{2}, A_{3} \cdots, Q\left(\bigcup_{n=1}^{\infty} A_{n}\right) \equiv \sum_{n=1}^{\infty} Q\left(A_{n}\right) \end{aligned}\).
	\[ \begin{aligned}
		Q\left(\bigcup_{n=1}^{\infty} A_{n}\right) &= P\left(\bigcup_{n=1}^{\infty} A_{n} \mid B\right) \\
		&=P\left(\left(\bigcup_{n=1}^{\infty} A_{n}\right) \cap B\right) / P(B) \quad (A_{n} \text{ are disjointed.}) \\
		&= \sum_{n=1}^{\infty} \frac{P\left(A_{n} \cap B\right)}{P(B)} \\
		&= \sum_{n=1}^{\infty} Q\left(A_{n}\right).
	\end{aligned}
	\]
\end{enumerate}

Thus, \((\Omega, \mathcal{F}, Q)\) is a probability space.

b)

For any C, set \(Q(C)>0\).
\[
Q(A \mid C)=\frac{Q(A \cap C) / P(B)}{Q(C)/ P(B)}=\frac{P(A \cap C \mid B) P(B)}{P(C \mid B) P(B)} = \frac{P(A \cap C \cap B)}{P(C \cap B) } = P(A \mid B \cap C).
\]

\medskip

%------------------------------------------------------%
%------------------------------------------------------%


\noindent
6. Prove the Continuity Theorem.

\noindent
Short answer,

\begin{enumerate}[label=\roman*)] % i), ii), iii), ...
	\item Check \textbf{M}onotone \textbf{i}ncreasing \textbf{s}equence of events. \\
	If \(A_{1} \subset A_{2} \subset A_{3} \subset \cdots\) s.t. \(A_{n} \uparrow A\), We need to show \(P(A)=\lim _{n \rightarrow \infty} P\left(A_{n}\right)\). Because \(A=\bigcup_{n=1}^{\infty} A_{n}\), We define \(B_{1}=A_{1}, \quad B_{2}=A_{2} \backslash A_{1}, \quad B_{3}=A_{3} \backslash A_{2}, \cdots,\)
	\[
	\begin{aligned}
		&\Rightarrow A & &= \bigcup_{n=1}^{\infty} B_{n}, \\
		&\Rightarrow P(A) & &=P\left(\bigcup_{n=1}^{\infty} B_{n}\right)=\sum_{n=1}^{\infty} P\left(B_{n}\right). \quad (B_{n}\text{'s are disjoint.) }
	\end{aligned}
	\]
	Because \(\quad B_{n}=A_{n} \backslash A_{n-1} \quad \) and \( \quad A_{n-1} \backslash A_{n}\).
	\[
	\begin{aligned}
		&\Rightarrow P\left(B_{n}\right) && = P\left(A_{n}\right)- P\left(A_{n-1}\right) \\
		&\Rightarrow \sum_{i=1}^{\infty} P\left(B_{n}\right)&& =\lim _{n \rightarrow \infty} P\left(A_{n}\right)-P\left(A_{0}\right)
	\end{aligned}
	\]
	Assume \(A_{0}=\emptyset\), \[\quad P(A)=\lim _{n \rightarrow \infty} P\left(A_{n}\right).\]
	\item Check \textbf{M}onotone \textbf{d}ecreasing \textbf{s}equence of events. \\
	Suppose \(D_{1} \supset D_{2} \supset D_{3} \cdots\), \(\begin{aligned} D=\bigcap_{n=1}^{\infty} D_{n} , \quad D_{n} \downarrow D \end{aligned}\).
	\[\begin{aligned}
		&\Rightarrow  \quad  D_{n}^{C}& & \uparrow & & D^{C}, \\
		&\Rightarrow \quad P\left(D^{C}\right)& & = & & \lim _{n \rightarrow \infty} P\left(D_{n}^{C}\right), \\
		&\Rightarrow \quad 1-P(D) & &= & &\lim _{n \rightarrow \infty}\left[1-P\left(D_{n}\right)\right], \\
		& \Rightarrow \quad P(D) & &= & &\lim _{n \rightarrow \infty} P\left(D_{n}\right).
	\end{aligned}
	\]
\end{enumerate}


%------------------------------------------------------%
%------------------------------------------------------%


\noindent
[Ref]: 

R. G. Bartle, D. R. Sherbert, ``Sets and Functions" in ``Introduction to Real Analysis", 3rd Edition, John Wiley and Sons, Inc. 2000. ch 1, pp 3.

W. Rudin, ``Basic Topology" in ``Principles of Mathematical Analysis", 3rd Edition, McGraw-Hill Inc. ch 2, pp 28.



\end{document}
